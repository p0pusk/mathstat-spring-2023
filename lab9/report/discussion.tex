\documentclass[./main.tex]{subfiles}

\begin{document}
\section{Обсуждение}

\subsection{Оценки исходной выборки} 

На основе полученных результатов можно сделать вывод, что верхние и нижние вершины оценок $\bm{J}_1$ совпадают с границами отображения на рис. \ref{pic:mode}.

\subsection{Мода и максимальная клика выборки} 

Полученная мода входит во все элементы выборки, что свидетельствуют о полной совместности выборки.

\subsection{Варьирование неопределенности изменений}

Полученная оценка постоянной $\beta$ почти совпала с нижней границей моды, вычисленной ранее. \\
Величина однородного расширения равна единице, что свидетельствует о полной совместности выборки.

\subsection{Коэффициент Жакара и относительная ширина моды}

Положительность коэффициента Жакара свидетельствует о перекрытии интервалов выборки. В данном случае, можно сделать вывод, что выборка совместна. Коэффициент Жакара и относительная ширина моды совпали, поскольку ширина пересечения интервалов и ширина моды выборки тоже совпали. \\

\newpage
\end{document}
