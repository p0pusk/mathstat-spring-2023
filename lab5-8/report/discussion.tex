\documentclass[main.tex]{subfiles}

\begin{document}
\section{Обсуждение}
\subsection{Выборочные коэффициенты корреляции}
\noindent Для двумерного нормального распределения дисперсии выборочных коэффициентов корреляции упорядочены следующим образом: $r \leq r_{S} \leq r_{Q}$; для смеси распределений получили обратную картину: $r_{Q} < r_{S} < r$.
\newline
\noindent Процент попавших элементов выборки в эллипс рассеивания (95$\%$-ная доверительная область) примерно равен его теоретическому значению (95$\%$).
\subsection{Оценки коэффициентов линейной регрессии}
\noindent По графикам можно сказать, что критерий наименьших квадратов точнее оценивает коэффициенты
линейной регрессии на выборке без возмущений. На выборке с редкими возмущениями 
критерий наименьших модулей графически показал себя более устойчивым. \\

\subsection{Проверка гипотезы о законе распределения генеральной совокупности. Метод хи-квадрат}

\noindent Заключаем, что гипотеза $H_{0}$ о нормальном законе распределения $N(x,\hat{\mu}, \hat{\sigma})$ на уровне значимости $\alpha = 0.05$ согласуется с выборкой для нормального распределения $N(x, 0, 1)$.
\\
Также видно, что для выборки сгенерированных по равномерному закону гипотеза $H_{0}$ оказалась принята. Однако для выборки, сгенерированной по закону Лапласа, основная гипотеза  $H_{0}$ не принята. \\ 
Это связано с тем, что на малых выборках критерий может ошибаться и для принятия гипотезы необходимо либо расширить выборку, либо понизить степень уверенности (т.е. увеличить уровень значимости $\alpha$).

\subsection{Доверительные интервалы для параметров распределения}
\begin{itemize}
	\item Генеральная характеристика $m$ = 0 полностью накрывается построенными доверительными интервалами, но $\sigma$ = 1 не накрывается, поскольку находится в интервале левой границы твина.
	\item Для более большой выборки доверительные интервалы являются более точными, т.е. меньшими по длине.
	\item Кроме того, при большом объеме выборки асимптотические и классические оценки практически совпадают.
	\item Для асимптотической оценки длины границ твинов в среднем получились больше чем для классической оценки.
\end{itemize}

\newpage
\end{document}
