\documentclass[main.tex]{subfiles}
\begin{document}
  
\section{Постановка задачи}

Имеется выборка данных с интервальной неопределенностью. Число отсчетов в выборке равно 200. Используется модель данных с  уравновешенным интервалом погрешности. \\

$\bm{x} = \ \stackrel{\circ}{x} + \ \boldsymbol{\bm{\epsilon}}$; \quad $\boldsymbol{\bm{\epsilon}} = [-\epsilon, \epsilon]$  для некоторого $\epsilon >0 $, \\


Здесь $\stackrel{\circ}{x}$ -- данные некоторого прибора, $\epsilon = 10 ^ {-4}$ -- погрешность прибора.

Необходимо \cite{b:task}:
\begin{itemize}
	\item Иллюстрировать данные выборки
	\item Построить диаграмму рассеяния
	\item Построить линейную регрессионную зависимость варьированием неопределенности изменений с расширением и без сужения интервалов
	\item Построить линейную регрессионную зависимость варьированием неопределенности изменений с расширением и сужением интервалов
	\item Произвести анализ регрессионных остатков
	\item Построить информационное множество по модели
	\item Проиллюстрирвоать коридор совместных зависимостей
	\item Построить прогноз вне области данных
\end{itemize}


\newpage

\end{document}
